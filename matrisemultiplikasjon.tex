\documentclass[12pt]{article}
\usepackage[utf8]{inputenc}
\usepackage{graphicx}
\usepackage{amsmath}
\usepackage{color}
\usepackage{comment}
\usepackage[table]{xcolor}

\newcommand\x{\times}
\newcommand\y{\cellcolor{blue!20}}
\newcommand\z{\cellcolor{red!20}}

\begin{document}
\section*{Matrisemultiplikasjon for nybyrjarar}

Matriser kan vera litt vrient i byrjinga.
Men det er ikkje så ille når ein fyrst forstår kva det går i.

Eg skal prøve å gje ein liten og \emph{forhåpentlegvis}
forståeleg intro\-duksjon til matriser og multiplikasjon av dei.

\subsection*{Matriser}

Kva er ei matrise? Det enklaste er å gje eit eksempel:

\begin{equation*}
X = \begin{bmatrix}
1 & 2 & 3 & 4 \\
5 & 6 & 7 & 8 \\
9 & 10 & 11 & 12
\end{bmatrix}
\end{equation*}
Her har me ei $3 \x 4$ matrise. 
Den består altså av 3 rader (``y-aksen'') og 4 kolonner (``x-aksen'').

Berre for å gjera det litt meir tydelig:
\begin{equation*}
\left[\begin{array}{cccc}
1 & 2 & 3 & 4 \\
\rowcolor{blue!20} 5 & 6 & 7 & 8 \\
9 & 10 & 11 & 12
\end{array}\right]
\end{equation*}
Her er \emph{rad} nummer 2 merka av i blått. Og motsvarande:
\begin{equation*}
\left[\begin{array}{cccc}
1 & 2 & \cellcolor{red!20} 3 & 4 \\
5 & 6 & \cellcolor{red!20} 7 & 8 \\
9 & 10 & \cellcolor{red!20} 11 & 12
\end{array}\right]
\end{equation*}
I matrisa over er \emph{kolonne} nummer 3 merka av i raudt.

Når ein referer element inne i matrisa, så nyttar ein fyrst rad og så kolonne.
Dette skriv ein som $X^{(rad)}_{kolonne}$.
Så f.eks. $X^{(2)}_1 = 5$.

For dei som har programmert litt kan ein merke seg at både rad og kolonne byrjar
på 1 (og ikkje på 0, slik det er i mange programmeringsspråk). Dette kan
variere avhengig av kva miljø du befinn deg i, men Octave er for eksempel 1-basert.

\subsection*{Vektor}

Ein vektor er eit spesialtilfelle av matriser.
Meir spesifikt er det ei matrise med vilkårlig antall
rader og nøyaktig ei kolonne. Den er med andre ord rad-basert.
Eit eksempel:
\begin{equation*}
\left[\begin{array}{c}
1 \\
2 \\
3 \\
4
\end{array}\right]
\end{equation*}
Dette kaller ein ein 4-dimensjonal vektor,
ein vektor med fire dimensjonar eller eventuelt ei $4 \x 1$ matrise.

\subsection*{Kolonnevektor}

Som du kanskje kan gjette deg til, så er dette det samme som ein vektor,
men den går ``andre vegen''. Dvs. det er ei matrise med nøyaktig ei rad
og vilkårlig antall kolonner. For eksempel:

\begin{equation*}
\left[\begin{array}{cccc}
1 & 2 & 3 & 4
\end{array}\right]
\end{equation*}

Dette er altså ein kolonnevektor, 
column based vector eller
berre column vector på engelsk.

\subsection*{Transpose}

Korleis kan ein gå mellom ein vanlig vektor og ein kolonnevektor?
Jo, ein kunne sjølvsagt skrive ein for loop for det, men ein har
ein matriseoperasjon for det. Transpose heiter den, og det skrives
på følgande måte:

\begin{equation*}
X = \left[\begin{array}{cccc}
1 \\
2 \\
3 \\
4
\end{array}\right]\text{, }
X^T = \left[\begin{array}{cccc}
1 & 2 & 3 & 4\\
\end{array}\right]
\end{equation*}
Skrivemåten er altså $X^T$ for å snu eller transpose matrisa $X$.
Om ein gjer to transpose så kjem ein tilbake til utgangspunktet, dvs.
$\left(X^T\right)^T = X$.

Transpose kan ein gjera på alle typer matriser, ikkje berre vektorer eller
kolonnevektorer, eksempelvis: 

\begin{equation*}
\left[\begin{array}{cccc}
\rowcolor{red!20}1 & 2 & 3 & 4 \\
5 & 6 & 7 & 8 \\
\rowcolor{blue!20}
9 & 10 & 11 & 12
\end{array}\right]^T = 
\left[\begin{array}{cccc}
\cellcolor{red!20}1 & 5 & \cellcolor{blue!20} 9 \\
\cellcolor{red!20}2 & 6 & \cellcolor{blue!20} 10 \\
\cellcolor{red!20}3 & 7 & \cellcolor{blue!20} 11 \\
\cellcolor{red!20}4 & 8 & \cellcolor{blue!20} 12
\end{array}\right]
\end{equation*}

Her kan du forhåpentlegvis sjå at rader har blitt til kolonner (``snudd 90 grader'').
Det som skjer litt meir formelt er at rad- og 
  kolonneplassering for kvart element byttar plass.

\subsubsection*{Addisjon og subtraksjon}

Dette er dei enklaste operasjonane ein kan gjere med matriser.
Det krever at begge matrisene har samme dimensjonar, og så er det eigentleg
berre å gjera addisjon eller subtraksjon element for element.
For eksempel:

\begin{equation*}
\left[\begin{array}{cccc}
2 & 3 \\
4 & 5 
\end{array}\right] + 
\left[\begin{array}{cccc}
6 & 7 \\ 8 & 9
\end{array}\right] =
\left[\begin{array}{cccc}
(2+6) & (3 + 7) \\
(4+8) & (5 + 9)
\end{array}\right] =
\left[\begin{array}{cccc}
8 & 10 \\
12 & 14
\end{array}\right]
\end{equation*}

Verre var det ikkje. Gankse rett fram.
Det er samme logikk for subtraksjon.

\subsection*{Matrise-vektor multiplikasjon}

Då er det på tide å byrje å gange saman nokon tall.
Og summere i samme sleng. Dette kallar ein matrisemultiplikasjon.
%og matrimultiplikasjon meir generelt, gjer: Ganger saman tall og summerer dei.
Rekkefølgen på operandane spelar ei rolle. Det vil seie at
$A \x B \neq B \x A$, der $A$ og $B$ er matriser.

Vi byrjar med den enklaste forma. Matrise-vektor multiplikasjon og
setter:
%La oss igjen ta eit eksempel:

\begin{gather*}
A = \left[ 
\begin{array}{ccc}
1 & 2 & 3 \\
4 & 5 & 6
\end{array}
 \right] \text{ , } B = 
\left[ 
\begin{array}{ccc}
7 \\
8 \\
9
\end{array}
\right]
\end{gather*}
Kva vert $A \x B$?
\begin{gather*}
\left[ 
\begin{array}{ccc}
1 & 2 & 3 \\
4 & 5 & 6
\end{array}
 \right] \x
\left[ 
\begin{array}{ccc}
7 \\
8 \\
9
\end{array}
\right] =\text{ ?}
\end{gather*}

Behold. Ein matrise-vektor multiplikasjon. Men veldig vanskelig er det ikkje.
%Det fyrste ein gjer --- \emph{mentalt} vel å merke --- er å snu eller transpose den andre matrisa slik at ein får følgande uttrykk:

\begin{comment}
\begin{gather*}
\left[ 
\begin{array}{ccc}
1 & 2 & 3 \\
4 & 5 & 6
\end{array}
 \right] \x
\left[ 
\begin{array}{ccc}
7 & 8 & 9
\end{array}
\right] =\text{ ?}
\end{gather*}
\end{comment}

Svar-matrisa, altså resultatet av multipliseringa, har antall rader lik
$A$ og antall kolonner lik $B$. Så no har ein altså
følgande form:

\begin{gather*}
\left[ 
\begin{array}{ccc}
1 & 2 & 3 \\
4 & 5 & 6
\end{array}
 \right] \x
\left[ 
\begin{array}{ccc}
7 \\ 8 \\ 9
\end{array}
\right] =
\left[ 
\begin{array}{ccc}
? \\ ?
\end{array}
\right]
\end{gather*}

Om eg så seier at ``$?$'' skal vera summen av produktet for kvar rad, 
kan du gjette kva det vert? Vel, det vert altså som følgande for rad 1:

\begin{gather*}
\left[ 
\begin{array}{ccc}
\rowcolor{blue!20}1 & 2 & 3 \\
4 & 5 & 6
\end{array}
 \right] \x
\left[ 
\begin{array}{ccc}
\cellcolor{blue!20}7 \\ 
\cellcolor{blue!20} 8 \\ 
\cellcolor{blue!20} 9
\end{array}
\right] =
\left[ 
\begin{array}{ccc}
\rowcolor{blue!20}(1*7) + (2*8) + (3*9) \\
? \\
%?
%(4*7) + (5*8) + (6*9) \\
\end{array}
\right]
\end{gather*}
Og tilsvarande for rad nummer to:
\begin{gather*}
\left[ 
\begin{array}{ccc}
1 & 2 & 3 \\
\rowcolor{red!20}
4 & 5 & 6
\end{array}
 \right] \x
\left[ 
\begin{array}{ccc}
\cellcolor{red!20}7 \\ 
\cellcolor{red!20} 8 \\ 
\cellcolor{red!20} 9
\end{array}
\right] =
\left[ 
\begin{array}{ccc}
(1*7) + (2*8) + (3*9) \\
\rowcolor{red!20}
(4*7) + (5*8) + (6*9) \\
\end{array}
\right]
\end{gather*}
Så det som skjer her er at ein berre ganger inn kolonna i $B$ med elementa i kvar
rad i $A$, og så summerer dei opp.

\begin{gather*}
\left[ 
\begin{array}{ccc}
1 & 2 & 3 \\
4 & 5 & 6
\end{array}
 \right] \x
\left[ 
\begin{array}{ccc}
7 \\ 8 \\ 9
\end{array}
\right] =
\left[ 
\begin{array}{ccc}
7 + 16 + 27 \\
28 + 40 + 54 \\
\end{array}
\right] =
\left[ 
\begin{array}{ccc}
50 \\
122 \\
\end{array}
\right]
\end{gather*}

Det var ikkje veldig vanskelig, var det vel?
Vel. Litt vanskelig var det kanskje. 
Ville det ikkje vore litt enklare å skrive $B$ som $\left[ \begin{array}{ccc}7 &
                                                                                 8
                                                                                 &
                                                                                   9
\end{array}\right]$? Slik at ein hadde fått følgande:

\begin{gather*}
\left[ 
\begin{array}{ccc}
1 & 2 & 3 \\
4 & 5 & 6
\end{array}
 \right] \x
\left[ 
\begin{array}{ccc}
7 & 8 & 9
\end{array}
\right] =
\left[ 
\begin{array}{ccc}
(1*7) + (2*8) + (3*9) \\
(4*7) + (5*8) + (6*9) \\
\end{array}
\right]
\text{\color{red} NB: Hypotetisk!}
\end{gather*}
Då er det litt enklare å sjå for seg korleis ganginga og summeringa 
kjem til å verta gjort, 
er det ikkje? Då berre ganger ein inn rad mot rad i operandane.
Du er neppe åleine om å kunne ynskje deg dette.
Og dette kan ein få til med transpose. Då er det berre å ta transpose på
den andre operanden:

\begin{gather*}
\left[ 
\begin{array}{ccc}
1 & 2 & 3 \\
4 & 5 & 6
\end{array}
 \right] \x
\left[ 
\begin{array}{ccc}
7 & 8 & 9
\end{array}
\right]^T =
\left[ 
\begin{array}{ccc}
(1*7) + (2*8) + (3*9) \\
(4*7) + (5*8) + (6*9) \\
\end{array}
\right]
\end{gather*}
Det er unektelig litt enklare å kunne jobbe med $A$ og $B$
som har likt antall kolonner og berre sjå for seg at rad for rad i $A$ vert
ganga inn med rad for rad i $B$. Dette oppnår ein altså med å bruke $A \x B^T$.
Ganske kult\footnote{Kult nok iallefall.}, eller kva?

\subsection*{Matrise-matrise multiplikasjon}

Så no kan du forhåpentlegvis matrise-vektor multiplikasjon.
Kva med matrise-matrise multiplikasjon? Altså:

\begin{gather*}
\left[ 
\begin{array}{ccc}
1 & 2 & 3 \\
4 & 5 & 6
\end{array}
 \right] \x
\left[ 
\begin{array}{ccc}
7 & 10 \\
8 & 11 \\
9 & 12 \\
\end{array}\right]
= \text{ ?}
\end{gather*}
Som før har ein at svar-matrisa har likt antall rader som fyrste operand, 
og likt antall kolonner som andre operand.

\begin{gather*}
\left[ 
\begin{array}{ccc}
1 & 2 & 3 \\
4 & 5 & 6
\end{array}
 \right] \x
\left[ 
\begin{array}{ccc}
7 & 10 \\
8 & 11 \\
9 & 12 \\
\end{array}\right]
= 
\left[ 
\begin{array}{ccc}
? & ? \\
? & ? \\
\end{array}\right]
\end{gather*}
Frå før veit me at me har som følgande
\begin{gather*}
\left[ 
\begin{array}{ccc}
\rowcolor{blue!20}
1 & 2 & 3 \\
4 & 5 & 6
\end{array}
 \right] \x
\left[ 
\begin{array}{ccc}
\cellcolor{blue!20}7 & 10 \\
\cellcolor{blue!20}8 & 11 \\
\cellcolor{blue!20}9 & 12 \\
\end{array}\right]
= 
\left[ 
\begin{array}{ccc}
\cellcolor{blue!20}50 & ? \\
? & ? \\
\end{array}\right]
\end{gather*}
og
\begin{gather*}
\left[ 
\begin{array}{ccc}
1 & 2 & 3 \\
\rowcolor{red!20}
4 & 5 & 6
\end{array}
 \right] \x
\left[ 
\begin{array}{ccc}
\cellcolor{red!20}7 & 10 \\
\cellcolor{red!20}8 & 11 \\
\cellcolor{red!20}9 & 12 \\
\end{array}\right]
= 
\left[ 
\begin{array}{ccc}
50 & ? \\
\cellcolor{red!20}
122 & ? \\
\end{array}\right]
\end{gather*}

Kan du klare å gjette deg til kva spørsmålteikna vert?
Ikkje? Ta deg ein luftepause på fem minutt.

Det vert altså som følgande
\begin{gather*}
\left[ 
\begin{array}{ccc}
\rowcolor{blue!20}
1 & 2 & 3 \\
4 & 5 & 6
\end{array}
 \right] \x
\left[ 
\begin{array}{ccc}
7 & \y 10 \\
8 & \y 11 \\
9 & \y 12 \\
\end{array}\right]
= 
\left[ 
\begin{array}{ccc}
50 & \y (1*10) + (2*11) + (3*12) \\
122 & ? \\
\end{array}\right]
\end{gather*}
Og tilsvarande får ein
\begin{gather*}
\left[ 
\begin{array}{ccc}
1 & 2 & 3 \\
\rowcolor{red!20}
4 & 5 & 6
\end{array}
 \right] \x
\left[ 
\begin{array}{ccc}
7 & \z 10 \\
8 & \z 11 \\
9 & \z 12 \\
\end{array}\right]
= 
\left[ 
\begin{array}{ccc}
50 & (1*10) + (2*11) + (3*12) \\
122 & \z (4*10) + (5*11) + (6*12) \\
\end{array}\right]
\end{gather*}
Som igjen vert
\begin{gather*}
\left[ 
\begin{array}{ccc}
1 & 2 & 3 \\
4 & 5 & 6
\end{array}
 \right] \x
\left[ 
\begin{array}{ccc}
7 & 10 \\
8 & 11 \\
9 & 12 \\
\end{array}\right]
= 
\left[ 
\begin{array}{ccc}
50 & 68 \\
122 & 167 \\
\end{array}\right]
\end{gather*}

Om ein igjen ynskjer å arbeide med ei transposed utgåve av dette, så får ein
\begin{gather*}
\left[ 
\begin{array}{ccc}
1 & 2 & 3 \\
4 & 5 & 6
\end{array}
 \right] \x
\left[ 
\begin{array}{ccc}
7 & 8 & 9\\
10 & 11 & 12
\end{array}\right]^T
= 
\left[ 
\begin{array}{ccc}
50 & 68 \\
122 & 167 \\
\end{array}\right]
\end{gather*}
Og her kan det vera verdt å leggje merke til at svar-matrisa veks med ei
kolonne for kvar rad som vert lagt til i den andre operanden som 
vert snudd eller transposed.

Så dette var eit forsøk på å gje ein uformell, intuitiv kjensle med
korleis ein gjer matrise-matrise multiplikasjon. Særlig vil eg anbefale
å tenke på at begge operander kan ha like mange kolonner, men at ein då
lyt gjere transpose på den andre operanden før ein ganger dei saman.

Lukke til med multipliseringa.

-- \emph{Ivar Refsdal}. refsdal.ivar@gmail.com


\begin{comment}
\begin{gather*}
\begin{bmatrix}1 \\ 2 \\ 3\end{bmatrix} * \begin{bmatrix}2\end{bmatrix} = 
\begin{bmatrix}1*2 \\ 2*2 \\ 3*2\end{bmatrix} =
\begin{bmatrix}2 \\ 4 \\ 6\end{bmatrix}
\end{gather*}
Som du kanskje har lagt merke til, så er det ingen summering heilt enno.
Men det kjem. Det kjem.

Her hadde ein altså ei matrise $3\times{}1$ som skal gangast med ei $1\times{}1$
matrise. Det er i praksis scalar multiplikasjon, altså at ein berre ganger inn
eit tall i kvart element. Kva om det hadde vore andre vegen? $B * A$?

\begin{gather*}
\begin{bmatrix}2\end{bmatrix} *
\begin{bmatrix}1 \\ 2 \\ 3\end{bmatrix} = \text{\color{red}Feil!}
\end{gather*}
Kvifor er det feil? Jo, fordi det er følgande krav til matrisene i ein
gangeoperasjon:
Antall kolonner i venstre matrise må vera lik antall rader i høgre matrise.
Meir spesifikt så tillater ein berre multiplikasjon av matriser som har
formen $(m\times{}n) * (n\times{}c)$. Og for dette eksempelet som vart feil
så hadde ein altså $(1\times{}\emph{1}) * (\emph{3}\times{}1)$, og sidan 
$1 \neq{} 3$ så kan ein ikkje gjennomføre multiplikasjonen.

Rader og kolonner. Det har eg ikkje eingong nevnt, 
og det går unektelig ofte litt i surr for meg sjølv. 
Kva er kva?

%\begin{figure*}
\begin{equation*}
  \text{To rader:}
  \left[\begin{array}{cccc}
    \rowcolor{red!20}
    \x  & \x  & \x & \x \\
    0   & \x  & \x & \x \\
    \rowcolor{blue!20}
    0   & 0   & \x & \x \\
    0   & 0   & 0  & \x \\
    a  &  b  & c &  d\\
  \end{array}\right]
  \text{og to kolonner: }
  \left[\begin{array}{cccc}
    \y \x  & \x  & \cellcolor{red!20} \x & \x \\
    \y 0   & \x  & \cellcolor{red!20} \x & \x \\
    \y 0   & 0   & \cellcolor{red!20} \x & \x \\
    \y 0   & 0   & \cellcolor{red!20} 0  & \x \\
    \y a  &  b  & \cellcolor{red!20} c &  d\\
  \end{array}\right]
\end{equation*}
%\caption{To kolonner markert}
%\end{figure*}

\subsection*{Matrisemultiplikasjon}

Okay, så gitt ein har følgande matrise:

\begin{gather*}
X = 
\begin{bmatrix}
x_1^{(1)} & x_2^{(1)} & x_3^{(1)} & x_4^{(1)} \\
x_1^{(2)} & x_2^{(2)} & x_3^{(2)} & x_4^{(2)} \\
x_1^{(3)} & x_2^{(3)} & x_3^{(3)} & x_4^{(3)}
\end{bmatrix}
\end{gather*}

Då er altså $X$ ei $3 \x 4$ matrise. Og då lyt den andre matrisa
vera $4 \x \emph{\text{something}}$.
La oss setje den til å vera $4 \x 1$:

\begin{gather*}
Y = 
\begin{bmatrix}
y_1^{(1)} \\
y_1^{(2)} \\
y_1^{(3)} \\
y_1^{(4)}
\end{bmatrix}
\end{gather*}
Kva vert så $X * Y$? Det vert intet mindre enn følgande:

\begin{gather*}
\begin{bmatrix}
x_1^{(1)} & x_2^{(1)} & x_3^{(1)} & x_4^{(1)} \\
x_1^{(2)} & x_2^{(2)} & x_3^{(2)} & x_4^{(2)} \\
x_1^{(3)} & x_2^{(3)} & x_3^{(3)} & x_4^{(3)}
\end{bmatrix} * 
\begin{bmatrix}
y_1^{(1)} \\
y_1^{(2)} \\
y_1^{(3)} \\
y_1^{(4)}
\end{bmatrix} = 
\begin{bmatrix}
x_1^{(1)}y_1^{(1)} +
x_2^{(1)}y_1^{(2)} +
x_3^{(1)}y_1^{(3)} +
x_4^{(1)}y_1^{(4)} \\
x_1^{(2)}y_1^{(1)} +
x_2^{(2)}y_1^{(2)} +
x_3^{(2)}y_1^{(3)} +
x_4^{(2)}y_1^{(4)} \\
x_1^{(3)}y_1^{(1)} +
x_2^{(3)}y_1^{(2)} +
x_3^{(3)}y_1^{(3)} +
x_4^{(3)}y_1^{(4)}
\end{bmatrix}
\end{gather*}
Det som eg pleier å gjera er å sjå for meg følgande:

\begin{equation*}
\left[\begin{array}{cccc}
\rowcolor{red!20}x_1^{(1)} & x_2^{(1)} & x_3^{(1)} & x_4^{(1)} \\
x_1^{(2)} & x_2^{(2)} & x_3^{(2)} & x_4^{(2)} \\
x_1^{(3)} & x_2^{(3)} & x_3^{(3)} & x_4^{(3)}
\end{array}\right] * 
\left[\begin{array}{cccc}
\rowcolor{blue!10}
y_1^{(1)} & y_1^{(2)} & y_1^{(3)} & y_1^{(4)}
\end{array}\right] =
\left[\begin{array}{cccc}
\rowcolor{green!20}
x_1^{(1)}y_1^{(1)} +
x_2^{(1)}y_1^{(2)} +
x_3^{(1)}y_1^{(3)} +
x_4^{(1)}y_1^{(4)} 
\end{array}\right]
\end{equation*}
Merk at dette er berre noko eg ser for meg. Men her er altså Y ``snudd''
eller ``transposed'' på engelsk.
Og når ein har gjort det, så er det berre å gange inn kvart element med
kvarandre
og summere dei opp. Ein ganger altså rada i X med ei kolonne i Y, og
tek summen av dette. Slik gjer ein for alle radene.

\begin{equation*}
\left[\begin{array}{cccc}
x_1^{(1)} & x_2^{(1)} & x_3^{(1)} & x_4^{(1)} \\
\rowcolor{red!20}
x_1^{(2)} & x_2^{(2)} & x_3^{(2)} & x_4^{(2)} \\
x_1^{(3)} & x_2^{(3)} & x_3^{(3)} & x_4^{(3)}
\end{array}\right] * 
\left[\begin{array}{cccc}
\rowcolor{blue!10}
y_1^{(1)} & y_1^{(2)} & y_1^{(3)} & y_1^{(4)}
\end{array}\right] =
\left[\begin{array}{cccc}
x_1^{(1)}y_1^{(1)} +
x_2^{(1)}y_1^{(2)} +
x_3^{(1)}y_1^{(3)} +
x_4^{(1)}y_1^{(4)} \\
\rowcolor{green!20}
x_1^{(2)}y_1^{(1)} +
x_2^{(2)}y_1^{(2)} +
x_3^{(2)}y_1^{(3)} +
x_4^{(2)}y_1^{(4)} \\
\end{array}\right]
\end{equation*}
Og for den siste rada i X:

\begin{equation*}
\left[\begin{array}{cccc}
x_1^{(1)} & x_2^{(1)} & x_3^{(1)} & x_4^{(1)} \\
x_1^{(2)} & x_2^{(2)} & x_3^{(2)} & x_4^{(2)} \\
\rowcolor{red!20}
x_1^{(3)} & x_2^{(3)} & x_3^{(3)} & x_4^{(3)}
\end{array}\right] * 
\left[\begin{array}{cccc}
\rowcolor{blue!10}
y_1^{(1)} & y_1^{(2)} & y_1^{(3)} & y_1^{(4)}
\end{array}\right] =
\left[\begin{array}{cccc}
x_1^{(1)}y_1^{(1)} +
x_2^{(1)}y_1^{(2)} +
x_3^{(1)}y_1^{(3)} +
x_4^{(1)}y_1^{(4)} \\
x_1^{(2)}y_1^{(1)} +
x_2^{(2)}y_1^{(2)} +
x_3^{(2)}y_1^{(3)} +
x_4^{(2)}y_1^{(4)} \\
\rowcolor{green!20}
x_1^{(3)}y_1^{(1)} +
x_2^{(3)}y_1^{(2)} +
x_3^{(3)}y_1^{(3)} +
x_4^{(3)}y_1^{(4)} \\
\end{array}\right]
\end{equation*}

Merk at $x_1^{(1)}y_1^{(1)} + x_2^{(1)}y_1^{(2)} + x_3^{(1)}y_1^{(3)} +
x_4^{(1)}y_1^{(4)}$ let seg redusere til eitt tall. Slik sett kan ein
skrive resultatmatrisa på følgande måte:

\begin{equation*}
\left[\begin{array}{cccc}
        \sum_{i=1}^{n}{x_n^{(1)}y_1^{(n)}} \\
        \sum_{i=1}^{n}{x_n^{(2)}y_1^{(n)}} \\
        \sum_{i=1}^{n}{x_n^{(3)}y_1^{(n)}} \\
\end{array}\right]
\end{equation*}
Merk då at ein krever $m \x n$ og $n \x k$ for henholdsvis venstre og høgre matrise.
Merk også at Y-vektoren er gjort om til ein kolonnevektor --- og at dette berre
er for å gjera det enklare å sjå for seg korleis multiplikasjonen og summering 
vert gjort.
\end{comment}




\end{document}